\documentclass[12pt]{article}

\usepackage{a4wide}
\usepackage[utf8]{inputenc}

\usepackage{graphicx}

%\pagestyle{empty}

\parindent=0pt

\begin{document}

\section*{Inverse Discrete Fourier Transform}

In the previous lesson, we've implemented the Discrete Fourier Transform. Today, we'll implement its inverse called the Inverse Discrete Fourier Transform (IDFT).

We want to transform the frequency spectrum $F$ back to its spatial domain $f$, which is an image.
The computation of the IDFT can be easily implemented using following formula

\begin{equation}
    f(m, n) = \sum\limits_{k=0}^{M-1} \sum\limits_{l=0}^{N-1} F(k, l) \varphi_{k, l}(m, n)\,.
\end{equation}

The basis $\varphi_{k, l}$ is defined as

\begin{equation}
    \varphi_{k, l}(m, n) = \frac{1}{\sqrt{MN}} e^{i 2 \pi \left( \frac{mk}{M} + \frac{nl}{N} \right) }, k = 0, 1, \dots, M-1 \,\, \mathrm{a} \,\, l = 0, 1, \dots, N-1\,.
\end{equation}
Notice that argument of $e$ is positive, so it's different from the basis used in DFT.
\\
\\
To compute the basis, it's advantageous to use Eulers formula $e^{ix} = \cos( x ) + i \sin( x )$.
\\
\\
Don't forget that elements in the matrix $F$ are complex, so is the basis. Think for a while, what will be the result.
\\
\\
\textbf{Hint:} Use \texttt{double} data type as last time.

\end{document}

